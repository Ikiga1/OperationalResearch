\documentclass[1pt]{article}
\usepackage {tikz}
\usetikzlibrary {arrows,automata}
\usepackage{amssymb}
\usepackage{listings}
\usepackage{pdfpages}
\usepackage{caption}
\usepackage[nottoc]{tocbibind}
\usepackage[T1]{fontenc}
\usepackage{graphicx}

\usepackage{amsmath}

\def\X#1{$#1$ &\tt\string#1}
\newcommand\tab[1][1cm]{\hspace*{#1}}

\makeatletter
\let\@fnsymbol\@arabic
\makeatother
\title{\huge \textrm{Supermarket and Clerks}}

\date{\textrm{Politecnico di Milano, \\15 November 2018}}


\begin{document}

	\pagenumbering{gobble}
	\begin{titlepage}
		\maketitle
	\end{titlepage}

	\pagenumbering{arabic}

	\newpage
	\section{Introduction to the Problem}
		We are given a problem: we have to minimize the cost of clerks working in a supermarket using linear programming, find the dual problem and give an interpretation of it.\\
		We are provided with: 
		\begin{itemize}
			\item $d = \{M, T, W, Th, F, S, Su\}$, the set of days in the week. 
			\item $b_d$, minimum number of clerks required in day $d$.
			\item $c_d$, cost of salary for the shift beginning in day $d$.
		\end{itemize}
		We also know that each clerk has to work 5 consecutive days and then has 2 days of rest.
	\section{Solution}
		I thought to model the problem as follows.\\
		To start with I introduce a variable 
		\begin{align}
		x &= \begin{bmatrix}
           x_{1} \\
           x_{2} \\
           \vdots \\
           x_{7}
         \end{bmatrix}
		\end{align}
		that represents the number of clerks starting to work on the $i-th$ day. From now on I consider $b$ and $c$ as the column vectors with the meaning specified on the previous section.\\
		It is possible to formulate the problem as:
		\begin{itemize}
			\item min ($\boldsymbol{x}\boldsymbol{c^T}$)
			\item $\boldsymbol{Ax} \geqslant \boldsymbol{b}$
			\item $\boldsymbol{x} \geqslant 0$
		\end{itemize}
		Since the number of clerks working on day $i$ is equal to the sum of all the clerks that started to work in the 4 days before $i$ and on $i$ itself, the matrix $A$ is defined as:
		\begin{align}
		A &= \begin{bmatrix}
				1 & 0 & 0 & 1 & 1 & 1 & 1 \\
				1 & 1 & 0 & 0 & 1 & 1 & 1 \\
				1 & 1 & 1 & 0 & 0 & 1 & 1 \\
				1 & 1 & 1 & 1 & 0 & 0 & 1 \\
				1 & 1 & 1 & 1 & 1 & 0 & 0 \\
				0 & 1 & 1 & 1 & 1 & 1 & 0 \\
				0 & 0 & 1 & 1 & 1 & 1 & 1 \\
             \end{bmatrix}
		\end{align}
		\newpage
		To get the dual problem, I define the $y$ (column vector) variable and following the table reported on the slides for the correspondence of primal-dual, the dual problem result as:
		\begin{itemize}
			\item max ($\boldsymbol{yb^T}$)
			\item $\boldsymbol{A^Ty} \leqslant \boldsymbol{c}$
			\item $\boldsymbol{y} \geqslant 0$
		\end{itemize}
		We can imagine $y$ to represent the cost of work of clerks on day $i$. Clerks wants to maximize the total profit they can make, given the budget of the supermarket on the day $i$ (and the number of working clerks on day $i$).

\end{document}